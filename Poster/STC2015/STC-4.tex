\documentclass[12pt]{article}

\usepackage{a4}
\usepackage{epsfig}
\usepackage{amsmath}
\usepackage{color}
\usepackage{psfig}
\usepackage{pstricks}
\usepackage[absolute]{textpos}

\setlength{\parindent}{0cm}
\setlength{\parskip}{2ex}
\setlength{\textwidth}{19cm}
\setlength{\textheight}{29cm}
\setlength{\headheight}{-3cm}
\setlength{\oddsidemargin}{-2.cm}

\pagestyle{empty}

\setlength{\TPHorizModule}{1cm}
\setlength{\TPVertModule}{1cm}

\newrgbcolor{darkred}{0.8784 0.573 0.522 }
\newrgbcolor{lightred}{1 0.863 0.733}
\newrgbcolor{llred}{1.  1. 1.}
\newrgbcolor{hellb}{.6 .6 1}
\newrgbcolor{gray94}{.94 .94 .94}
\newrgbcolor{blblue}{.2 .2 .7}
\definecolor{medsgrin}{rgb}{.235, .702, .443}
\definecolor{dunko}{rgb}{1,1,.3}

\begin{document}

\psframebox[fillcolor=white,fillstyle=solid,linewidth=3pt,
linecolor=blblue,framearc=.2]{
\begin{minipage}[c][27cm]{19cm}
\vspace{15mm}
{ \sffamily \Huge
\begin{center}
\textbf{ \color{blblue} Summary}
\end{center} }
\vspace{-4mm}
\sf
\vspace{3mm}
\bigskip \bigskip \bigskip
{\Large \bf \sf Conclusions } \\
\begin{itemize}
\item Photoelectron spectrum can be reconstructed from a $\rho$-TDCI calculation
\item Improved dependence of the signal on the delay time vs. TDDFT/FISH$^{(5)}$
\item Dependence of the signal on the delay time can be explained from transition dipole moments
\item Advantages of this method:
\begin{itemize}
\item Description of the system is systematically improvable
\item Strong field excitations can be treated variationally (linear extension)
\item Photoionization, relaxation, and internal conversion are included
\end{itemize}
\item Disadvantages:
\begin{itemize}
\item HF/CIS description is insufficient
\item Better description of the experiment requires (D) calculation (more expensive)
\end{itemize}
\end{itemize} \bigskip \bigskip
{\Large \bf \sf Outlook } \\
\begin{itemize}
\item Improve description of the experiment with various density functionals $\rightarrow$ TDCIS/DFT
\item Improve description of photoionization, relaxation, and internal conversion
\item Reduce computational requirements $\rightarrow$ propagation in CSF space and resolution of identity
\end{itemize} \bigskip \bigskip
{ \sffamily \Huge
\begin{center}
\textbf{ \color{blblue} References}
\end{center} } \bigskip
\begin{enumerate}
\item T. Klamroth, \emph{Phys. Rev. B} \textbf{68}, 245421 (2003).
\item J. C. Tremblay, T. Klamroth, and P. Saalfrank, \emph{J. Chem. Phys.} \textbf{129}, 084302 (2008).
\item S. Klinkusch, P. Saalfrank, and T. Klamroth, \emph{J. Chem. Phys.} \textbf{131}, 114304 (2009).
\item J. C. Tremblay, S. Klinkusch, T. Klamroth, and P. Saalfrank, \emph{J. Chem. Phys.} \textbf{134}, 044311 (2011).
\item J. Stanzel, M. Neeb, W. Eberhardt, P. G. Lisinetskaya, J. Petersen, and R. Mitri\'c, \emph{Phys. Rev. A} \textbf{85}, 013201 (2012).
\item F. Furche, R. Ahlrichs, P. Weis, C. Jacob, S. Gilb, T. Bierweiler, and M. M. Kappes, \emph{J. Chem. Phys.} {\bfseries 117}, 6982 (2002).
\end{enumerate}


\vspace{1cm}
\end{minipage} }
\vspace{3cm}
\end{document}
