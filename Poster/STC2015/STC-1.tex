\documentclass[12pt]{article}

\usepackage{a4}
\usepackage{epsfig}
\usepackage{amsmath, amssymb}
\usepackage{color}
\usepackage{psfig}
\usepackage{pstricks,pst-all}
\usepackage[absolute]{textpos}
\newcommand{\bra}[1]{\langle#1|}
\newcommand{\ket}[1]{|#1\rangle}
\newcommand{\braket}[2]{\langle#1|#2\rangle}
\newcommand{\bracket}[3]{\langle#1|#2|#3\rangle}
\newcommand{\back}[1]{\!\!\!\!\!\!\!\!\!\!\!\! #1}
\newcommand{\braaket}[2]{\langle#1||#2\rangle}
\newcommand{\vex}[1]{\underline{#1}}
\newcommand{\supop}[1]{\hat{\hat{#1}}}
\newcommand{\exv}[1]{\left\langle #1 \right\rangle}
\newcommand{\hhat}[1]{\hat{\hat{#1}}}
\newcommand{\ents}[0]{\widehat{=}}
\newcommand{\att}[0]{\frac{\hbar}{E_h}}
\newcommand{\lcolx}[0]{blblue}

\setlength{\parindent}{0cm}
\setlength{\parskip}{2ex}
\setlength{\textwidth}{19cm}
\setlength{\textheight}{29cm}
\setlength{\headheight}{-3cm}
\setlength{\oddsidemargin}{-2.cm}

\pagestyle{empty}

\setlength{\TPHorizModule}{1cm}
\setlength{\TPVertModule}{1cm}

\newrgbcolor{darkred}{0.8784 0.573 0.522 }
\newrgbcolor{lightred}{1 0.863 0.733}
\newrgbcolor{llred}{1.  1. 1.}
\newrgbcolor{hellb}{.6 .6 1}
\newrgbcolor{gray94}{.94 .94 .94}
\newrgbcolor{blblue}{.2 .2 .7}
\definecolor{medsgrin}{rgb}{.235, .702, .443}
\definecolor{dunko}{rgb}{1,1,.3}

\begin{document}

\psframebox[fillcolor=white,fillstyle=solid,linewidth=3pt,
linecolor=\lcolx,framearc=.2]{
\begin{minipage}[c][27cm]{19cm}
\vspace{15mm}
{ \sffamily \Huge
\begin{center}
    \textbf{ \color{\lcolx} Aim}
\end{center} }
\vspace{-4mm}
\sf
%\vspace{3mm} %\bigskip
\begin{itemize}
    \item Reconstruction of a time-resolved photoelectron spectrum of an Au$_7^{-}$ cluster
    \begin{itemize}
	\item Excitation of the cluster to higher states or to a wavepacket using a first laser pulse
	\item Excitation of the relaxed wavepacket to the ionization continuum using a second pulse
	\item Varying delay times between the two pulses
	\item Reconstruction of the spectrum from the kinetic energies of the ionized species
    \end{itemize}
\item Improvement of the temporal behaviour of the signal
\item Possibility to improve the signal by taking higher excitations into account and to use every \\ arbitrarily chosen laser field
\end{itemize}
{ \sffamily \Huge
\begin{center}
    \textbf{ \color{\lcolx} The $\boldsymbol{\rho}$-TDCI method$^{(1,2)}$}
\end{center} }
\vspace{-4mm}
\sf
\vspace{3mm} \bigskip
{\Large \bf \sf Theory}
\begin{itemize}
\item Goal: Solution of the Liouville-von-Neumann equation \\
    \vspace{-0.3cm}
\begin{displaymath}
\hspace{-1cm}
 \psframebox[fillcolor=yellow, fillstyle=solid, linewidth=2pt, framearc=.3]{
     \frac{\partial \hat{\rho} (t)}{\partial t} = -i \left[\hat{H}_{el}, \hat{\rho}\right] +i \left[\vex{\hat{\mu}} \vex{F} (t), \hat{\rho} (t)\right] + \supop{\mathcal{L}}_D \hat{\rho} (t) 
}
\end{displaymath}
within the space of CI eigenfunctions $\ket{n}$ with time-dependent expansion coefficients $\rho_{nm} (t)$ \\
    \vspace{-0.3cm}
\begin{displaymath}
\hspace{-1cm}
 \psframebox[fillcolor=yellow, fillstyle=solid, linewidth=2pt, framearc=.3]{
     \hat{\rho} (t) = \sum\limits_{nm} \rho_{nm} (t) \ket{n}\bra{m}
}
\end{displaymath}
\end{itemize}
{\Large \bf \sf Photoionization$^{(3,4)}$}
\begin{itemize}
    \item Photoionization rate of an electronic state $\ket{n}$ is calculated as \\
    \vspace{-0.3cm}
\begin{displaymath}
\hspace{-1cm}
 \psframebox[fillcolor=yellow, fillstyle=solid, linewidth=2pt, framearc=.3]{
 I_n = \begin{cases} 0 &\text{ , if } E_n < IP \\ \sum\limits_{a,r} |D_{a,n}^r|^2 \Omega_a^r &\text{ , if } E_n \ge IP \end{cases}
}
\end{displaymath} 
with the ionization rate of a configuration state function \\
    \vspace{-0.3cm}
\begin{displaymath}
    \hspace{-1cm}
    \psframebox[fillcolor=yellow, fillstyle=solid, linewidth=2pt, framearc=.3]{
    \Omega_a^r = \begin{cases} 0 &\text{ , if } \varepsilon_r \le 0 \\ \frac{\sqrt{\varepsilon_r}}{d} &\text{ , if } \varepsilon_r > 0 \end{cases}
    }
\end{displaymath}
\end{itemize}
{\Large \bf \sf Relaxation}
\begin{itemize}
\item Non dipole interactions and internal conversion is described using \\
    \vspace{-0.3cm}
\begin{displaymath}
\hspace{-1cm}
 \psframebox[fillcolor=yellow, fillstyle=solid, linewidth=2pt, framearc=.3]{
     \Gamma_{m\rightarrow n} = \frac{|\exv{H}_{NAC}|^2}{\hbar^2 (\omega_{nm} + \omega_c)^2}
     }
\end{displaymath}
with $|\exv{H}_{NAC}|^2$ as a system-dependent parameter and $\omega_c$ as the Matsubara frequency in an Ohmic expression of the phonon density of states \\
    \vspace{-0.3cm}
\begin{displaymath}
\hspace{-1cm}
 \psframebox[fillcolor=yellow, fillstyle=solid, linewidth=2pt, framearc=.3]{
     J (\omega) = \eta \omega e^{-\omega/\omega_c}
     }
\end{displaymath}
\end{itemize}
\vspace{1cm}
\end{minipage} }
\vspace{3cm}
\end{document}
