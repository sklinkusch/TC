\documentclass[12pt]{article}

\usepackage{a4}
\usepackage{epsfig}
\usepackage{amsmath}
\usepackage{color}
\usepackage{psfig}
\usepackage{pstricks}
\usepackage[absolute]{textpos}
\usepackage{units}

\setlength{\parindent}{0cm}
\setlength{\parskip}{2ex}
\setlength{\textwidth}{19cm}
\setlength{\textheight}{29cm}
\setlength{\headheight}{-3cm}
\setlength{\oddsidemargin}{-2.cm}

\pagestyle{empty}

\setlength{\TPHorizModule}{1cm}
\setlength{\TPVertModule}{1cm}

\newcommand{\lcolx}[0]{blblue}

\newrgbcolor{darkred}{0.8784 0.573 0.522 }
\newrgbcolor{lightred}{1 0.863 0.733}
\newrgbcolor{llred}{1.  1. 1.}
\newrgbcolor{hellb}{.6 .6 1}
\newrgbcolor{gray94}{.94 .94 .94}
\newrgbcolor{blblue}{.2 .2 .7}
\definecolor{medsgrin}{rgb}{.235, .702, .443}
\definecolor{dunko}{rgb}{1,1,.3}

\begin{document}

\psframebox[fillcolor=white,fillstyle=solid,linewidth=3pt,
linecolor=\lcolx,framearc=.2]{
\begin{minipage}[c][27cm]{19cm}
\vspace{15mm}
{ \sffamily \Huge
\begin{center}
\textbf{ \color{\lcolx} Probe excitation}
\end{center} }
\vspace{-4mm}
\sf
\vspace{3mm}
{\Large \bf \sf Signal reconstruction } \\
\begin{minipage}[!t]{9cm}
\begin{center}
\epsfig{file=rho-trpes.eps, width=9cm, clip=}
\end{center}
\end{minipage} 
\begin{minipage}[!t]{9.5cm}
\begin{itemize}
\item $N$ electron system consists of $m$ nonionizing and $(N-m)$ ionizing states
\item coherences among ionizing states are neglected, populations are kept
\item only populations of $(N-1)$ electron system are kept, no coherences
\item coupling between $N$ and $(N-1)$ electron system via ionization
\item coupling within $N$ electron system via elastic and inelastic processes
\end{itemize}
\end{minipage}
{\Large \bf \sf TRPES signal } \\
\begin{minipage}[!t]{9.5cm}
\begin{center}
\epsfig{file=au7-trpesMOD.eps, width=8cm, clip=}
\end{center}
\end{minipage}
\begin{minipage}[!t]{9.5cm}
\begin{center}
\epsfig{file=au7-trpes3dfilteredMOD.eps, width=8cm, clip=}
\end{center}
\end{minipage} \\ 
\begin{minipage}[!t]{17.5cm}
\begin{itemize}
\item Signals at kinetic energies lower than $\unit[1.3]{eV}$ decrease for higher delay times (as in experiment)
\item Signals at kinetic energies higher than $\unit[1.3]{eV}$ increase for higher delay times (artifact)
\end{itemize}
\end{minipage} \\
{\Large \bf \sf Interpretation } \\
\begin{minipage}[t]{18.5cm}
\begin{itemize}
\item Difference signal is calculated as \\
\end{itemize}
\begin{displaymath}
\hspace{-1cm}
\psframebox[fillcolor=yellow,fillstyle=solid,linewidth=2pt,framearc=.3]{
\Delta S_{probe} (E_{kin}) = S_{probe} (E_{kin}, t_{delay} = \unit[6]{ps}) - S_{probe} (E_{kin}, t_{delay} = \unit[100]{fs})
}
\end{displaymath}
\end{minipage} \\*[0.5cm]
\begin{minipage}[!t]{9cm}
\begin{center}
\epsfig{file=au7-diffkin2.eps, width=8cm, clip=}
\end{center}
\end{minipage}
\begin{minipage}[!t]{9.5cm}
\begin{itemize}
\item Positive signals arise from excitations from the ground state
\item Negative signals arise from excitations from the first excited state
\item Below $\unit[1.3]{eV}$ stronger signals on negative side
\item Above $\unit[1.3]{eV}$ stronger signals on positive side
\item Reason is a high transition dipole moment from the ground state to ionization continuum
\end{itemize}
\end{minipage} \\
\bigskip
\vspace{1cm}
\end{minipage} }
\vspace{3cm}
\end{document}
